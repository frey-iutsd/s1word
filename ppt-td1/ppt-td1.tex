% !TeX root = ppt-td1.tex
\input{../prelude.tex}
\usepackage[a4paper, margin=1in]{geometry}
\usepackage{hyperref}
\begin{document}
\title{{\bf \large{PowerPoint TD1}}}
\author{}%\email{bonzom@univ-paris13.fr}
%\affiliation{{\footnotesize LIPN, Institut Galilée, Sorbonne Paris Cité, 99 avenue Jean-Baptiste Clément, 93430 Villetaneuse, France, EU}}
%\affiliation{{\footnotesize IUT Saint-Denis, Département GEA, Place du 8 mai 1945 - 93206 Saint-Denis, France, EU}}
\date{\small 12 novembre 2024}
\maketitle

\begin{questions}


\uplevel{L'objectif de cet exercice est de reproduire la présentation dans le
fichier \texttt{saint-denis.pdf}. Suivez les instructions, puis déposez un
fichier \emph{PowerPoint} sur Teams que vous nommerez
\texttt{votrenom-ppt-td1.ppt}. Si vous ne pouvez pas accéder à Teams,
envoyez-moi votre fichier par e-mail.}

\medskip

\question Créez une nouvelle présentation \emph{PowerPoint} vierge et
choisissez le thème \emph{Vue} (\texttt{Conception --> Thèmes --> Vue}),
troisième variante.

\question Sur la première diapositive, mettez ``La commune de Saint Denis''
comme titre et votre nom en tant que sous-titre.

\question Diapositive 2 : 
\begin{parts}
\item 
Utilisez la disposition \texttt{Titre et contenue} et mettez le titre
``Administration''. 
\item Dans la zone de texte principale, insérez une liste à
puces comme dans le modèle. Vous pouvez copier-coller le contenu de l'infobox de
la page \emph{Wikipedia} sur Saint
Denis\footnote{\url{https://fr.wikipedia.org/wiki/Saint-Denis_(Seine-Saint-Denis)}}.
Il convient d'utiliser le mode de collage `Conserver uniquement le texte'
(\texttt{Accueil --> Coller Conserver uniquement le texte}).
\item
Appliquez le style \texttt{Effet discret -- Marron, 3 accentué} à la zone de
texte (\texttt{Format de forme --> Styles de formes}).
\end{parts}
\question Diapositive 3 :
\begin{parts}
\item
Utilisez la disposition \texttt{Image avec légende}. 
\item Pour l'image, insérez
la photo de l'hôtel de ville de Saint-Denis, que vous téléchargerez de la page
Wikipédia%
\footnote{\url{https://fr.wikipedia.org/wiki/H\%C3\%B4tel_de_ville_de_Saint-Denis_(Seine-Saint-Denis)}}
(cliquez d'abord sur l'image, puis faites un clic droit dans la vue agrandie
pour l'enregistrer).
\item Ajoutez titre et sous-titre comme dans le modèle.
\end{parts}
\question Diapositive 4 :
\begin{parts}
\item Utilisez la disposition \texttt{Deux contenus} et mettez ``Géographie''
pour le titre. 
\item A gauche, insérez le texte de la section \texttt{Géographie -->
Localisation} de la page Wikipédia, en deux puces. 
\item À droite, insérez la carte téléchargée de la même section. Appliquez le
style \texttt{Perspective relâchée, blanc} à l'image (\texttt{Format de l'image
--> Styles d'image}). 
\item Sous l'image, insérez une zone de texte (\texttt{Insertion --> Texte -->
Zone de texte}) et mettez la description.
\end{parts}
\question Diapositive 5 : 
\begin{parts}
\item Utilisez la disposition \texttt{Titre et contenu} et mettez ``Le
logement à Saint-Denis en 2018'' comme titre.
\item Dans la zone de contenu principale, insérez un tableau et ajoutez les
donnés que vous trouverez dans la section correspondante sur la page Wikipedia.
\item Cochez la case \texttt{Première colonne} dans l'onglet \texttt{Création de
tableau}, groupe \texttt{Options de style de tableau}, et choisissez le style
\texttt{Style moyen 3 -- Accentuation 3}.
\end{parts}
\question Diapositive 6 : 
\begin{parts}
\item Utilisez la disposition \texttt{Deux contenus} et mettez le titre
``Histoire de Saint Denis''.
\item A gauche, insérez au moins trois puces avec des événements historiques
tirés de la page Wikipédia.
\item À droite, insérez une image historique de Saint-Denis trouvée sur la même
page.
\item Sous l'image, insérez une zone de texte avec une explication.
\item Choisissez des styles qui vous plaisent pour la zone de texte et l'image.
\end{parts}
\item Dernière diapositive :
\begin{parts}
\item Utilisez la disposition \texttt{Vide}, insérez une zone de texte et
écrivez ``{Merci pour votre attention !}''.
\item Changez la police en \texttt{Palatino Linotype} de taille 54.
\item Choisissez un style qui vous plait sous \texttt{Format de forme --> Styles
WordArt}.
\item Centrez la zone de texte horizontalement et verticalement sur la
diapositive avec \texttt{Format de forme --> Aligner --> Centrer} et
\texttt{Format de forme --> Aligner --> Aligner au milieu}.
\end{parts}
\end{questions}
\end{document}
